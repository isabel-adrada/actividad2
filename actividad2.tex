% Options for packages loaded elsewhere
\PassOptionsToPackage{unicode}{hyperref}
\PassOptionsToPackage{hyphens}{url}
%
\documentclass[
]{article}
\usepackage{amsmath,amssymb}
\usepackage{iftex}
\ifPDFTeX
  \usepackage[T1]{fontenc}
  \usepackage[utf8]{inputenc}
  \usepackage{textcomp} % provide euro and other symbols
\else % if luatex or xetex
  \usepackage{unicode-math} % this also loads fontspec
  \defaultfontfeatures{Scale=MatchLowercase}
  \defaultfontfeatures[\rmfamily]{Ligatures=TeX,Scale=1}
\fi
\usepackage{lmodern}
\ifPDFTeX\else
  % xetex/luatex font selection
\fi
% Use upquote if available, for straight quotes in verbatim environments
\IfFileExists{upquote.sty}{\usepackage{upquote}}{}
\IfFileExists{microtype.sty}{% use microtype if available
  \usepackage[]{microtype}
  \UseMicrotypeSet[protrusion]{basicmath} % disable protrusion for tt fonts
}{}
\makeatletter
\@ifundefined{KOMAClassName}{% if non-KOMA class
  \IfFileExists{parskip.sty}{%
    \usepackage{parskip}
  }{% else
    \setlength{\parindent}{0pt}
    \setlength{\parskip}{6pt plus 2pt minus 1pt}}
}{% if KOMA class
  \KOMAoptions{parskip=half}}
\makeatother
\usepackage{xcolor}
\usepackage[margin=1in]{geometry}
\usepackage{longtable,booktabs,array}
\usepackage{calc} % for calculating minipage widths
% Correct order of tables after \paragraph or \subparagraph
\usepackage{etoolbox}
\makeatletter
\patchcmd\longtable{\par}{\if@noskipsec\mbox{}\fi\par}{}{}
\makeatother
% Allow footnotes in longtable head/foot
\IfFileExists{footnotehyper.sty}{\usepackage{footnotehyper}}{\usepackage{footnote}}
\makesavenoteenv{longtable}
\usepackage{graphicx}
\makeatletter
\def\maxwidth{\ifdim\Gin@nat@width>\linewidth\linewidth\else\Gin@nat@width\fi}
\def\maxheight{\ifdim\Gin@nat@height>\textheight\textheight\else\Gin@nat@height\fi}
\makeatother
% Scale images if necessary, so that they will not overflow the page
% margins by default, and it is still possible to overwrite the defaults
% using explicit options in \includegraphics[width, height, ...]{}
\setkeys{Gin}{width=\maxwidth,height=\maxheight,keepaspectratio}
% Set default figure placement to htbp
\makeatletter
\def\fps@figure{htbp}
\makeatother
\setlength{\emergencystretch}{3em} % prevent overfull lines
\providecommand{\tightlist}{%
  \setlength{\itemsep}{0pt}\setlength{\parskip}{0pt}}
\setcounter{secnumdepth}{-\maxdimen} % remove section numbering
\usepackage{multicol}
\usepackage{longtable}
\setlength{\columnsep}{1cm}
\usepackage{booktabs}
\usepackage{array}
\usepackage{float}
\usepackage{longtable}
\usepackage{multirow}
\usepackage{wrapfig}
\usepackage{colortbl}
\usepackage{pdflscape}
\usepackage{tabu}
\usepackage{threeparttable}
\usepackage{threeparttablex}
\usepackage[normalem]{ulem}
\usepackage{makecell}
\usepackage{xcolor}
\ifLuaTeX
  \usepackage{selnolig}  % disable illegal ligatures
\fi
\usepackage{bookmark}
\IfFileExists{xurl.sty}{\usepackage{xurl}}{} % add URL line breaks if available
\urlstyle{same}
\hypersetup{
  pdftitle={Actividad 2},
  hidelinks,
  pdfcreator={LaTeX via pandoc}}

\title{Actividad 2}
\author{Adrada Isabel, De la Peña Juan, Terán Federico, Troncoso
Samuel\\
Pontificia Universidad Javeriana Cali}
\date{}

\begin{document}
\maketitle

\begin{multicols}{2}

\section{Resumen}
...

\section{Key words}
...

\section{Introducción}
...

\section{Métodos}
...

En primer lugar, se generaron aleatoriamente 1000 números entre 1 y 100000, los cuáles representan la población objetivo del presente estudio mediante la función sample.


``` r
poblacion <- sample(0:1000, 1000, replace = TRUE)
```



El promedio de la población objetivo de los 1000 números se obtuvo utilizando la función mean sobre el vector datos generado anteriormente.

``` r
promedio <- mean(datos)
```

Para generar una tabla de frecuencia de la población, se realizaron k  intérvalos, donde k es igual a 10, con un ancho de (Max - Min)/k por intérvalo. El conteo de la cantidad de datos dentro de un determinado conforma la frecuencia absoluta, se presenta además la frecuencia relativa, frecuencia absoluta acumulada y frecuencia relativa acumulada.

\section{Resultados}



\end{multicols}

\begin{longtable}[]{@{}lcccc@{}}
\caption{Frecuencia de los datos}\tabularnewline
\toprule\noalign{}
Intervalo & F.Absoluta & F.Relativa & F.Abs.Acum & F.Rel.Acum \\
\midrule\noalign{}
\endfirsthead
\toprule\noalign{}
Intervalo & F.Absoluta & F.Relativa & F.Abs.Acum & F.Rel.Acum \\
\midrule\noalign{}
\endhead
\bottomrule\noalign{}
\endlastfoot
{[}0, 100{]} & 105 & 0.10 & 105 & 0.10 \\
(100, 200{]} & 109 & 0.11 & 214 & 0.21 \\
(200, 300{]} & 113 & 0.11 & 327 & 0.33 \\
(300, 400{]} & 97 & 0.10 & 424 & 0.42 \\
(400, 500{]} & 95 & 0.10 & 519 & 0.52 \\
(500, 600{]} & 108 & 0.11 & 627 & 0.63 \\
(600, 700{]} & 100 & 0.10 & 727 & 0.73 \\
(700, 800{]} & 82 & 0.08 & 809 & 0.81 \\
(800, 900{]} & 95 & 0.10 & 904 & 0.90 \\
(900, 1000{]} & 96 & 0.10 & 1000 & 1.00 \\
\end{longtable}

\begin{multicols}{2}


\begin{center}
\includegraphics[width=\linewidth]{figura1.png}
\end{center}
Figura 1. Distribución de los datos.


\section{Análisis de resultados}
...

\section{Conclusiones}
...

\section{Referencias}
...

\end{multicols}

\end{document}
